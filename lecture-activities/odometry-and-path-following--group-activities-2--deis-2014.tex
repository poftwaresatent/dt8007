\documentclass[a4paper]{article}

\usepackage{booktabs,amsmath}
\usepackage[margin=30mm]{geometry}


\begin{document}
\title{
  \large
  DT8007 Design of Embedded Intelligent Systems\\
  \Large
  Group Activities Odometry and Path Following (Part II)}
\author{Roland Philippsen}
\maketitle

\noindent
Form groups of 3-4 students and work together on the following tasks.

\section*{Pose Feedback Control}

\begin{enumerate}

\item
  Compute the polar errors for the following situations.
  
  \begin{center}
    \begin{tabular}{lll}
      \toprule
      case & pose           & goal                                      \\
           & $(x,y,\theta)$ & $(x_\text{g},y_\text{g},\theta_\text{g})$ \\
      \midrule
      a & $( 0,  0, 0)            $ & $( 1,  1,  \frac{ \pi}{2})$ \\
      b & $( 1,  1, 0)            $ & $( 0,  0, -\frac{3\pi}{4})$ \\
      c & $(-1, -1, \frac{\pi}{4})$ & $( 1,  1,  \frac{-\pi}{4})$ \\
      d & $(-1, -1, \frac{\pi}{4})$ & $(-1, -1, -\frac{ \pi}{2})$ \\
      \bottomrule
    \end{tabular}
  \end{center}
  
\item
  For each of the cases above, what would be the desired speed $(v,\omega)$ for the following set of gains?
  \[
  k_\rho = 3,\; k_\delta = -1.5,\; k_\gamma = 8
  \]
  
\item
  Design a method for ensuring that the desired velocities $(v,\omega)$ never exceed given maxima, i.e.\ $|v|\leq v_\text{max}$ and $|\omega|\leq \omega_\text{max}$.
  Think about cases where either $v$ or $\omega$ violates their upper bound individually, as well as the case where both violate their bounds at the same time.
  What are the aspects that you have to take into account?
  Sketch an algorithm that adds such bound handling to the pose feedback controller.
  
\end{enumerate}

\section*{Spline Paths}

\begin{enumerate}
  
\item
  Given the following 6 control points, make a sketch of what the resulting uniform cubic B-spline $q(\lambda)$ looks like.
  Then, compute $q(\lambda)$ for $\lambda\in\{0, 1.5, 2.5\}$.
  
  \begin{center}
    \begin{tabular}{lllllll}
      \toprule
      case & $Q_0$ & $Q_1$ & $Q_2$ & $Q_3$ & $Q_4$ & $Q_5$ \\
      \midrule
      a & $( 0, 0)$ & $(1,  0)$ & $(2, 0)$ & $(3, -1)$ & $(3, 0)$ & $(3, 1)$ \\
      b & $(-1, 0)$ & $(0, -6)$ & $(1, 0)$ & $(2,  6)$ & $(3, 5)$ & $(4, 4)$ \\
      \bottomrule
    \end{tabular}
  \end{center}
  
\item
  Develop the formula for computing the tangent to a uniform cubic B-spline.
  Graphically determine the tangent to the curves you sketched for the previous exercise.
  Then compute the tangent for the same values of $\lambda$ and verify your computations using your sketch.

\item
  What is the angle between the tangent to the curve and the X-axis?

\item
  Develop the formula which determines how fast $\lambda$ has to change in order to make the point $q(\lambda)$ move with a desired velocity along the spline.
  
\end{enumerate}

\end{document}
